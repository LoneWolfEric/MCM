
% !Mode:: "TeX:UTF-8"
%% This is file `mcmthesis-demo.tex',
%% generated with the docstrip utility.
%%
%% The original source files were:
%%
%% mcmthesis.dtx  (with options: `demo')
%%
%% -----------------------------------
%%
%% This is a generated file.
%%
%% Copyright (C)
%%     2010 -- 2015 by Zhaoli Wang
%%     2014 -- 2016 by Liam Huang
%%     2017         by Zhuang Ma
%%     2018美赛,为了迈思数模内部学员更快的掌握CTEX,由迈思数模站长对本模板进行调整
%% This work may be distributed and/or modified under the
%% conditions of the LaTeX Project Public License, either version 1.3
%% of this license or (at your option) any later version.
%% The latest version of this license is in
%%   http://www.latex-project.org/lppl.txt
%% and version 1.3 or later is part of all distributions of LaTeX
%% version 2005/12/01 or later.
%%
%% This work has the LPPL maintenance status `maintained'.
%%
%% The Current Maintainer of this work is Liam Huang.
%% 上述内容为介绍改模板的背景

\documentclass{mcmthesis}
\mcmsetup{CTeX = true,   % 使用 CTeX 套装时,设置为 true
        tcn = O Prize Group, problem = B,
        sheet = true, titleinsheet = true, keywordsinsheet = true,
        titlepage = true, abstract = false}
%设置摘要页格式,一般按照该设置就行,true表示选择,false表示不选择,需要修改控制号和选择的题目
\usepackage{palatino}

%\usepackage{lipsum} %添加lipsum宏包,就是随机生成一段文本,可以不使用

\usepackage[UTF8, nocap]{ctex} %如果想使用中文输入的话,可以增加该宏包
\usepackage{amsmath}

\title{We Will Get O Prize(Not U Prize)}
%\author{\small \href{http://www.latexstudio.net/}
% {\includegraphics[width=7cm]{mcmthesis-logo}}}
%\date{\today}

 %正文摘要和控制页摘要名字修改
%\def\abstractname{Abstract}
%\def\sheetsummaryname{Summary}



\begin{document}

 %控制页摘要内容
\begin{sheetsummary}
%\lipsum[1]
%lipsum是显示一段无意义的文字
%Ru guo ni bu xiang xian shi zhe duan wen zi, qing ba \textit{lipsum} shanchu.\\
%如果你不想显示上面的那段文字,那就把 \textit{lipsum}删掉
%获得更多的latex教材,请添加latex交流群,或者关注迈思数模微信公众号:shumohome
%并回复“LATEX资料”
%美赛LATEX模板交流群:193607493
hello world
\end{sheetsummary}

% %正文摘要内容
%\begin{abstract}
%\lipsum[2]
%\end{abstract}

 %关键词
\begin{keywords}
hello; world
\end{keywords}

\maketitle
 % Generate the Table of Contents, if it's needed.
 \tableofcontents
 \newpage


\section{Introduction}
    \subsection{Background}
     In past decades, we’ve got plenty of researches about toll plaza design and operation model which were applied in our life, such as queuing theory, finite state machine princi-ple, and cellular automata model. Actually,  the designs were suitable.
      On 5 Oct. 2016 The New York Times reported that “In an effort to reduce congestion, tollbooths will be eliminated at all Metropolitan Transportation Authority bridges and tun-nels next year, and replaced with automatic tolling, Gov. Andrew M. Cuomo announced on Wednesday.”
      With the popularity of ETC and automated tollbooths, we need a new model to stimu-late traffic including accident prevention, throughput and cost.
      More than this, there will be more and more autonomous (self-driving) vehicles added to the traffic mix which make the traffic more complex than ever.
      So to accommodate this change, the point is to determine if there are better solutions (shape, size, and merging pattern) than any in common use.



\section{Literature review}

         A statistical test recommends that arrivals at the toll plaza conform to the standard assump-tion of a Poisson process with exponential interarrival times.This is well supported in the literature (Hasofer, 1964; Schwartz, 1974; Grassman, 1980; Green, 1985; Blackwell, 1988).


    \section{Assumptions}
        \begin{enumerate}
          \item There are three options for each vehicle arriving at the toll plaza and each car entered the shortest path of the queue.
          \item All the toll booths are same except charge method (width, construction cost e.t.c )
          \item The arrival process is Poisson
        \end{enumerate}


\section{Statement of our Model}

    \subsection{Definition}
        \begin{table}
          \centering
                \begin{tabular}{ll}
                \hline
                h  &  Conventional (human-staffed) tollbooths \\
                a  &  Exact-change (automated) tollbooths \\
                e  &  Electronic toll collection booths \\
                B$_{i}$  &  Number of types i tollbooths \\
                b$_{i}$  &  Number of type i tollbooths to open, where a \\
                B  &  Number of tollbooths \\
                L  &  Number of main lanes \\
                l$_{i}$  &  Lower bound for the number of type i lanes to open \\
                u$_{i}$  &  Upper bound for the number of type i lanes to open \\
                $\lambda _{i}$ & Mean arrival rate for lane type i, where \\
                \hline
                \end{tabular}
            \caption{Definition}
        \end{table}

        \begin{itemize}
              \item h--Conventional (human-staffed) tollbooths
              \item a--Exact-change (automated) tollbooths
              \item e--Electronic toll collection booths
              \item B$_{i}$--Number of types i tollbooths
              \item b$_{i}$--Number of type i tollbooths to open, where
              \item B--Number of tollbooths
              \item L--Number of main lanes
              \item l$_{i}$--Lower bound for the number of type i lanes to open
              \item u$_{i}$--Upper bound for the number of type i lanes to open
              \item $\lambda_{i}$--Mean arrival rate for lane type i, where
              \item $\lambda$--Mean total arrival rate of vehicles at the toll plaza, i.e., the number of arrivals per unit time
              \item $\mu_{i}$--Mean service rate for a type i tollbooth, i.e., the number of service completions per unit time
              \item $\sigma_{i}$--Standard deviation of service time for a type i tollbooth
              \item W--Mean total waiting time in the queue for all arrivals at the toll plaza
              \item c$_{i}$--The rate of the operating cost of a type i lane
              \item d--The rate of the operating cost of a type i lane
              \item c$_{o}$--The total operating costs at the toll plaza per unit time
              \item c$_{w}$--The total user-waiting costs at the toll plaza per unit time
              \item Z--The sum of total operating and user-waiting costs at the toll plaza per unit time
              \item l--the length of buffer segment
              \item w--the width of booth
              \item y$_{i}$--Traffic accident prediction
        \end{itemize}










    \begin{itemize}
    \item minimizes the discomfort to the hands, or
    \item maximizes the outgoing velocity of the ball.
    \end{itemize}
    We focus exclusively on the second definition.

    \begin{itemize}
    \item the initial velocity and rotation of the ball,
    \item the initial velocity and rotation of the bat,
    \item the relative position and orientation of the bat and ball, and
    \item the force over time that the hitter hands applies on the handle.
    \end{itemize}
    \begin{itemize}
    \item the angular velocity of the bat,
    \item the velocity of the ball, and
    \item the position of impact along the bat.
    \end{itemize}
    \emph{center of percussion} [Brody 1986],

    \begin{Theorem} \label{thm:latex}
    \LaTeX
    \end{Theorem}
    \begin{Lemma} \label{thm:tex}
    \TeX .
    \end{Lemma}
    \begin{proof}
    The proof of theorem.
    \end{proof}

    \subsection{Other Assumptions}

    \begin{itemize}
    \item
    \item
    \item
    \item
    \end{itemize}



\section{Analysis of the Problem}
\begin{figure}[h]
%\small
\centering
\includegraphics[width=12cm]{mcmthesis-aaa.eps}
\caption{Figure example 1} \label{fig:aa}
\end{figure}


Figure \ref{aa}.

\begin{figure}[h]
\begin{minipage}[h]{0.5\linewidth}
\centering
\includegraphics[width=0.8\textwidth]{0.jpg}
\caption{Figure example 2}
\end{minipage}
\begin{minipage}[h]{0.5\linewidth}
\centering
\includegraphics[width=0.8\textwidth]{0.jpg}
\caption{Figure example 3}
\end{minipage}
\end{figure}



\begin{equation}
a^2 \label{aa}
\end{equation}

\[
  \begin{pmatrix}{*{20}c}
  {a_{11} } & {a_{12} } & {a_{13} }  \\
  {a_{21} } & {a_{22} } & {a_{23} }  \\
  {a_{31} } & {a_{32} } & {a_{33} }  \\
  \end{pmatrix}
  = \frac{{Opposite}}{{Hypotenuse}}\cos ^{ - 1} \theta \arcsin \theta
\]


\[
  p_{j}=\begin{cases} 0,&\text{if $j$ is odd}\\
  r!\,(-1)^{j/2},&\text{if $j$ is even}
  \end{cases}
\]



\[
  \arcsin \theta  =
  \mathop{{\int\!\!\!\!\!\int\!\!\!\!\!\int}\mkern-31.2mu
  \bigodot}\limits_\varphi
  {\mathop {\lim }\limits_{x \to \infty } \frac{{n!}}{{r!\left( {n - r}
  \right)!}}} \eqno (1)
\]

\section{Calculating and Simplifying the Model  }


$\mathop{A}_{ij}$ and $A_{ij}$ 不一样

$\left(123 \right)$ and $(123)$ and (123) 不一样
\section{The Model Results}


\section{Validating the Model}

CTEX中最繁琐的是表格输入,不过数学建模竞赛中,一般要求输入三线表,这样统一格式的情况下,对于输入表格就简单一点了。\\
\\
双斜杠是强制换行,在矩阵,表格,大括号的公式常用。\\
关于表格的合并,见下一部分。给出的例子。
\begin{center}
\begin{tabular}{c|cclcrcc}
\hline
Year & theta & $S_1^-$ & $S_2^-$ & $S_3^-$ & $S_4^+$ & $S_5^+$ & $S_6^+$ \\%表格标题
\hline
2016 & 1      & 0      & 0 & 0.0001 & 0      & 0      & 0 \\
2017 & 0.9997 & 0.0555 & 0 & 0.2889 & 0.1844 & 0.463  & 0 \\
2018 & 0.9994 & 0      & 0 & 0.0012 & 0.3269 & 0.7154 & 0 \\
2019 & 0.9993 & 0      & 0 & 0      & 0.4325 & 1.0473 & 0 \\
2020 & 0.9991 & 0      & 0 & 0      & 0.5046 & 1.2022 & 0 \\
2021 & 0.999  & 0      & 0 & 0      & 0.5466 & 1.2827 & 0 \\
2022 & 0.9989 & 0.0017 & 0 & 0.3159 & 0.562  & 1.2995 & 0 \\
2023 & 0.9989 & 0      & 0 & 0.0109 & 0.5533 & 1.2616 & 0 \\
2024 & 0.9989 & 0      & 0 & 0      & 0.5232 & 1.1769 & 0 \\
2025 & 0.9989 & 0      & 0 & 0.1009 & 0.4738 & 1.0521 & 0 \\
2026 & 0.9991 & 0      & 0 & 0      & 0.4071 & 0.8929 & 0 \\
2027 & 0.9992 & 0.0004 & 0 & 0.1195 & 0.3248 & 0.7042 & 0 \\
2028 & 0.9994 & 0.0164 & 0 & 0.046  & 0.2287 & 0.4902 & 0 \\
2029 & 0.9997 & 0      & 0 & 0.0609 & 0.12   & 0.2545 & 0 \\
2030 & 1      & 0      & 0 & 0      & 0      & 0      & 0 \\
\hline
\end{tabular}
\end{center}

\section{Conclusions}


\begin{center}
\begin{tabular}{c|cc}
\hline
年份 & \multicolumn{2}{c}{指标}\\
\hline
2017 & 0.9997 & 0.0555 \\
2018 & 0.9994 & 0      \\
2019 & 0.9993 & 0      \\
\hline
\end{tabular}
\end{center}


\begin{table}[h]
  \centering
  \begin{tabular}{c|cc}
\hline
年份 & \multicolumn{2}{c}{指标}\\
\hline
2017 & 0.9997 & 0.0555 \\
2018 & 0.9994 & 0      \\
2019 & 0.9993 & 0      \\
\hline
\end{tabular}
  \caption{NAME}\label{SIGN}
\end{table}

Let's to see Table \ref{SIGN}.


\begin{minipage}{0.5\linewidth}
\begin{tabular}{|c|c|c|}
\hline
\multicolumn{2}{|c|}{\multirow{2}{*}{合并}}&测试\\
\cline{3-3}
\multicolumn{2}{|c|}{}& 0.9997  \\
\hline
2019 & 0.9993 & 0 \\
\hline
\end{tabular}
\end{minipage}
\begin{minipage}{0.5\linewidth}
\begin{tabular}{c|ccc}
\hline
年份 & \multicolumn{3}{c}{指标}\\
\hline
\multirow{3}{*}{合并}&2017 & 0.9997 & 0.0555 \\
&2018 & 0.9994 & 0      \\
&2019 & 0.9993 & 0      \\
\hline
\end{tabular}
\end{minipage}





\section{A Summary}


写入你的中文
如果你不想显示上面的那段文字,那就把lipsum删掉

获得更多的latex教材,请添加latex交流群,或者关注迈思数模微信公众号:shumohome
并回复“LATEX资料”
美赛LATEX模板交流群:193607493
美赛LATEX模板交流群:193607493

\section{Evaluate of the Mode}

美赛获得更多的latex教材,请添加latex交流群,或者关注迈思数模微信公众号:shumohome
并回复“LATEX资料”
美赛LATEX模板交流群:193607493
\section{Strengths and weaknesses}

获得更多的latex教材,请添加latex交流群,或者关注迈思数模微信公众号:shumohome
并回复“LATEX资料”
美赛LATEX模板交流群:193607493

\subsection{Strengths}
\begin{itemize}
\item \textbf{Applies widely}\\
This  system can be used for many types of airplanes, and it also
solves the interference during  the procedure of the boarding
airplane,as described above we can get to the  optimization
boarding time.We also know that all the service is automate.
\item \textbf{Improve the quality of the airport service}\\
Balancing the cost of the cost and the benefit, it will bring in
more convenient  for airport and passengers.It also saves many
human resources for the airline. \item \textbf{}
\end{itemize}

\begin{thebibliography}{99}
\bibitem{1} D.~E. KNUTH   The \TeX{}book  the American
Mathematical Society and Addison-Wesley
Publishing Company , 1984-1986.
\bibitem{2}Lamport, Leslie,  \LaTeX{}: `` A Document Preparation System '',
Addison-Wesley Publishing Company, 1986.
\bibitem{3}\url{http://www.latexstudio.net/}
\bibitem{4}\url{http://www.chinatex.org/}
\end{thebibliography}

\begin{appendices}

\section{First appendix}



Here are simulation programmes we used in our model as follow.\\

\textbf{\textcolor[rgb]{0.98,0.00,0.00}{Input matlab source:}}
\lstinputlisting[language=Matlab]{./code/mcmthesis-matlab1.m}

\section{Second appendix}

some more text \textcolor[rgb]{0.98,0.00,0.00}{\textbf{Input C++ source:}}
\lstinputlisting[language=C++]{./code/mcmthesis-sudoku.cpp}

\end{appendices}
\end{document}

%%
%% This work consists of these files mcmthesis.dtx,
%%                                   figures/ and
%%                                   code/,
%% and the derived files             mcmthesis.cls,
%%                                   mcmthesis-demo.tex,
%%                                   README,
%%                                   LICENSE,
%%                                   mcmthesis.pdf and
%%                                   mcmthesis-demo.pdf.
%%
%% End of file `mcmthesis-demo.tex'.
